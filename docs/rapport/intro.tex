\chapter*{Introduction}
\addcontentsline{toc}{chapter}{Introduction}

\paragraph{}
Afin de valider les compétences acquises concernant la création d'application sur terminaux mobiles, les étudiants par groupe de quatre ont chacun reçu
un projet Android.

\paragraph{}
Notre sujet consiste à créer un \textbf{GPS piéton}. Ce GPS offrira plusieurs possibilités, deux principales
seront de retrouver des points d'intérêts, que nous appelleront par la suite \textbf{POI}, sur une carte ou à l'aide de la
réalité augmentée. Nous avons décidé d'utiliser Google API pour notre application. Ce choix est dû principalement au fait que certains 
membres du groupe ont déjà une premiere expérience sur ces API. 

\paragraph{}
Trois grandes idées de travails sont ressorties la récupérations de POI via le service Google, la réalité augmentée en utilisant la caméra, 
et le travail de calcul d'itinéraire sur la Map.

La première partie consiste à créér une classe permettant la communication entre notre application et la base de donnée à Google. La seconde utilise la
caméra afin d'afficher les POI sur l'écran, et la dernière partie utilise les informations reçues pour les afficher de manière classique sur une carte.
